\documentclass[12pt,a4paper]{report}

\usepackage[utf8]{inputenc}
\usepackage[T1]{fontenc}
\usepackage[french]{babel}
\usepackage{geometry}
\usepackage{graphicx}
\usepackage{hyperref}
\usepackage{setspace}
\usepackage{lipsum}

\newcommand{\Nom}{Jean Dupont}
\newcommand{\Age}{35}
\newcommand{\Profession}{Ingénieur}
\newcommand{\ScoreEpuisement}{45}
\newcommand{\ScoreDetachement}{30}


% Configuration des marges
\geometry{top=2.5cm, bottom=2.5cm, left=2.5cm, right=2.5cm}

% Titre
\title{Rapport de Diagnostic \\ Burn-out}
\author{Nom du Professionnel}
\date{\today}

\begin{document}

\maketitle

\tableofcontents
\newpage

\chapter*{Introduction}
\addcontentsline{toc}{chapter}{Introduction}
Ce rapport présente l'évaluation des réponses fournies par une personne dans le cadre d’un questionnaire destiné à identifier les signes d’un éventuel burn-out. Les résultats permettent de dresser un diagnostic préliminaire et de proposer des recommandations adaptées.

\chapter{Informations générales}

\section{Identité de la personne évaluée}
\begin{itemize}
    \item **Nom** : \Nom
    \item **Âge** : \Age
    \item **Profession** : \Profession
    \item **Date de l'évaluation** : [Insérer la date]
\end{itemize}

\section{Objectif de l'évaluation}
Le questionnaire avait pour but d'évaluer :
\begin{itemize}
    \item L’épuisement émotionnel ;
    \item Le niveau de détachement vis-à-vis du travail ;
    \item La perception de l’efficacité personnelle.
\end{itemize}

\chapter{Analyse des résultats}

\section{Résumé des réponses au questionnaire}
\begin{itemize}
    \item **Score d'épuisement émotionnel** : [Insérer le score]
    \item **Score de détachement** : [Insérer le score]
    \item **Score d'efficacité personnelle** : [Insérer le score]
\end{itemize}

\section{Interprétation des résultats}
Sur la base des scores obtenus :
\begin{itemize}
    \item Un score élevé en épuisement émotionnel indique [analyse du résultat].
    \item Un score élevé en détachement vis-à-vis du travail suggère [analyse du résultat].
    \item Une faible efficacité personnelle reflète [analyse du résultat].
\end{itemize}

Les réponses qualitatives aux questions ouvertes révèlent également :
\begin{itemize}
    \item [Insérer une synthèse des réponses importantes].
\end{itemize}

\chapter{Diagnostic et évaluation clinique}

\section{Conclusion du diagnostic}
À la lumière des réponses et de leur interprétation, la personne évaluée présente :
\begin{itemize}
    \item [Insérer un diagnostic, par ex. : des signes clairs de burn-out / un risque modéré de burn-out / aucun signe significatif de burn-out].
\end{itemize}

\section{Facteurs contributifs identifiés}
Les principaux facteurs de risque identifiés incluent :
\begin{itemize}
    \item [Exemple : charge de travail excessive, conflits interpersonnels, absence de reconnaissance].
\end{itemize}

\chapter{Recommandations}

\section{Mesures à court terme}
\begin{itemize}
    \item **Repos** : Encourager la prise de congés ou des pauses régulières.
    \item **Accompagnement** : Consulter un psychologue ou un coach professionnel.
    \item **Réduction du stress** : Pratiquer des techniques de relaxation (méditation, sport).
\end{itemize}

\section{Mesures à long terme}
\begin{itemize}
    \item **Réévaluation des priorités professionnelles** : Ajuster la charge de travail.
    \item **Renforcement des compétences en gestion du stress** : Participer à des formations spécifiques.
    \item **Suivi régulier** : Planifier des évaluations périodiques.
\end{itemize}

\chapter*{Conclusion}
\addcontentsline{toc}{chapter}{Conclusion}
Le diagnostic indique que [insérer synthèse du diagnostic]. La mise en œuvre des recommandations proposées est essentielle pour prévenir l'aggravation des symptômes et favoriser une meilleure qualité de vie, tant sur le plan professionnel que personnel.

\appendix
\chapter{Annexe : Questionnaire rempli}
\label{annex:questionnaire}

Les réponses au questionnaire sont les suivantes :
\begin{itemize}
    \item Question 1 : [Insérer réponse].
    \item Question 2 : [Insérer réponse].
    \item Question 3 : [Insérer réponse].
\end{itemize}

\end{document}
